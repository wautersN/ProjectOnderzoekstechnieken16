\documentclass[conference]{IEEEtran}
\begin{document}
\title{Project BlackJack}
\author{
\IEEEauthorblockN{Jelle Biebaut, Daan Delva, Jari Meire, Niels Wauters}
\IEEEauthorblockA{HoGent\\Onderzoekstechnieken\\Groep16}}
\maketitle

\IEEEpeerreviewmaketitle

\section{Inleiding}
Blackjack is een card game met eenvoudige spelregels. Een speler probeert om een score te behalen zo dicht mogelijk bij 21 zonder dat hij erboven gaat, tegen een dealer. Zolang dat geen van beide boven de 21 gaan, wint de hoogste score. Maar achter dit spel zijn een verschillende strategie\"en bedacht om zo veel mogelijk te kunnen winnen. In deze paper gaan we uitzoeken welke strategie het beste is om zo veel mogelijk spelletjes te winnen. We gaan gebruik maken van simulatie software om zo rapper tot de resultaten te komen.
 
\hfill Maart 28, 2015

\subsection{Onderzoeksvraag}
Wat is de beste strategie?



\section{Strategie\"en}
In het spel blackjack zijn er verschillende strategie\"en uitgevonden om zo veel mogelijk te kunnen winnnen. In onze paper gaan we ze niet allemaal bespreken, maar enkel de strategie\"en die het meest gebruikt worden.


\subsection{Basic strategy}

Dit is een strategie die gekopie\"ert is van ervaren gokkers. In onze simulatie maken we gebruik van 4 of meerdere deck. Als je blackjack hebt dan krijg je 3:2 uitbetaling. De dealer stands als hij een soft 17 heeft en het is toegestaan om een double te doen achter dat je gesplit hebt. 

\begin{table}[ht]
\tiny
\begin{tabular}{|l|l|l|l|l|l|l|l|l|l|l|}
\hline

{Player hand} & \multicolumn{10}{c|}{Dealer's face-up card}     \\ \cline{2-11} 
                             & 2  & 3  & 4  & 5  & 6  & 7  & 8  & 9  & 10 & A  \\ \hline
\multicolumn{11}{|c|}{Hard totals}                                             \\ \hline
17-20       								& S  & S  & S  & S  & S  & S  & S  & S  & S  & S  \\ \hline
16                           & S  & S  & S  & S  & S  & H  & H  & H  & H  & H  \\ \hline
15                           & S  & S  & S  & S  & S  & H  & H  & H  & H  & H  \\ \hline
13-14                        & S  & S  & S  & S  & S  & H  & H  & H  & H  & H  \\ \hline
12                           & H  & H  & S  & S  & S  & H  & H  & H  & H  & H  \\ \hline
11                           & DH & DH & DH & DH & DH & DH & DH & DH & DH & H  \\ \hline
10                           & DH & DH & DH & DH & DH & DH & DH & DH & H  & H  \\ \hline
9                            & H  & DH & DH & DH & DH & H  & H  & H  & H  & H  \\ \hline
5-8                          & H  & H  & H  & H  & H  & H  & H  & H  & H  & H  \\ \hline
\multicolumn{11}{|c|}{Soft totals}                                             \\ \hline
                             & 2  & 3  & 4  & 5  & 6  & 7  & 8  & 9  & 10 & A  \\ \hline
A,8-A,9                      & S  & S  & S  & S  & S  & S  & S  & S  & S  & S  \\ \hline
A,7                          & S  & DS & DS & DS & DS & S  & S  & H  & H  & H  \\ \hline
A,6                          & H  & DH & DH & DH & DH & H  & H  & H  & H  & H  \\ \hline
A,4-A,5                      & H  & H  & DH & DH & DH & H  & H  & H  & H  & H  \\ \hline
A,2-A,3                      & H  & H  & H  & DH & DH & H  & H  & H  & H  & H  \\ \hline
\multicolumn{11}{|c|}{Pairs}                                                   \\ \hline
                             & 2  & 3  & 4  & 5  & 6  & 7  & 8  & 9  & 10 & A  \\ \hline
A,A                          & SP & SP & SP & SP & SP & SP & SP & SP & SP & SP \\ \hline
10,10                        & S  & S  & S  & S  & S  & S  & S  & S  & S  & S  \\ \hline
9,9                          & SP & SP & SP & SP & SP & S  & SP & SP & S  & S  \\ \hline
8,8                          & SP & SP & SP & SP & SP & SP & SP & SP & SP & SP \\ \hline
7,7                          & SP & SP & SP & SP & SP & SP & H  & H  & H  & H  \\ \hline
6,6                          & SP & SP & SP & SP & SP & H  & H  & H  & H  & H  \\ \hline
5,5                          & DH & DH & DH & DH & DH & DH & DH & DH & H  & H  \\ \hline
4,4                          & H  & H  & H  & SP & SP & H  & H  & H  & H  & H  \\ \hline
2,2-3,3                      & SP & SP & SP & SP & SP & SP & H  & H  & H  & H  \\ \hline
\end{tabular}

\end{table}

Zoals hierboven afgebeeld heeft de speler de mogelijkheid om te Hitten(H),Stand(S),Double then hit(DH),Double then stand(DS) of om te splitten(SP). Om een voorbeeld te geven uit bovenstaande tabel, als de speler een hard total heeft van 15, dan moet hij standen als de dealer kaart lager is dan 7 of anders moet hij hitten.





\subsection{Simple strategy}

Deze strategie wordt vooraal gebruikt voor spelers die nog niet zoveel ervaring hebben met blackjack. De spelers zijn alleen maar gefocust op hun eigen kaarten en houden geen rekening met de kaarten van de dealer. Om als voorbeeld te geven, jij hebt 16 gedeeld gekregen en in dit geval stand je zonder rekening te houden met de dealer zijn kaart. Maar als de dealer een aas liggen heeft dan is het beter dat je hit omdat de kans groter is dat de dealer meer dan 16 gaan behalen.

\subsection{Counting cards}


\section*{Varia}




\begin{thebibliography}{1}

\bibitem{IEEEhowto:kopka}
H.~Kopka and P.~W. Daly, \emph{A Guide to \LaTeX}, 3rd~ed.\hskip 1em plus
  0.5em minus 0.4em\relax Harlow, England: Addison-Wesley, 1999.

\end{thebibliography}

\end{document}